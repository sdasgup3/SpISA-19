
\documentclass[a4paper,UKenglish,cleveref, autoref]{lipics-v2019}
%This is a template for producing LIPIcs articles. 
%See lipics-manual.pdf for further information.
%for A4 paper format use option "a4paper", for US-letter use option "letterpaper"
%for british hyphenation rules use option "UKenglish", for american hyphenation rules use option "USenglish"
%for section-numbered lemmas etc., use "numberwithinsect"
%for enabling cleveref support, use "cleveref"
%for enabling cleveref support, use "autoref"


%\graphicspath{{./graphics/}}%helpful if your graphic files are in another directory

\bibliographystyle{plainurl}% the mandatory bibstyle

\title{Formalized Instruction Decoder in \K For \ISA ISA} %TODO Please add

\titlerunning{x86-64 ISA semantics}%optional, please use if title is longer than one line

%\author{Sandeep Dasgupta}{University of Illinois at Urbana Champaign, USA \and \url{http://sdasgup3.web.engr.illinois.edu}}{sdasgup3@illinois.edu}{https://orcid.org/0000-0002-6483-2055}%TODO mandatory, please use full name; only 1 author per \author macro; first two parameters are mandatory, other parameters can be empty. Please provide at least the name of the affiliation and the country. The full address is optional

%\author{Sandeep Dasgupta}{University of Illinois at Urbana Champaign, USA \and \url{http://sdasgup3.web.engr.illinois.edu} }{sdasgup3@illinois.edu}{https://orcid.org/0000-0002-6483-2055}{}
\author{Sandeep Dasgupta}{University of Illinois at Urbana Champaign, USA \and \url{http://sdasgup3.web.engr.illinois.edu} }{sdasgup3@illinois.edu}{}{}
\author{Andrew H. Miranti}{University of Illinois at Urbana Champaign, USA}{miranti2@illinois.edu}{}{}


\authorrunning{S. Dasgupta and A.H. Miranti}%TODO mandatory. First: Use abbreviated first/middle names. Second (only in severe cases): Use first author plus 'et al.'

\Copyright{Sandeep Dasgupta and Andrew H. Miranti}%TODO mandatory, please use full first names. LIPIcs license is "CC-BY";  http://creativecommons.org/licenses/by/3.0/

\ccsdesc[100]{General and reference~General literature}
\ccsdesc[100]{General and reference}%TODO mandatory: Please choose ACM 2012 classifications from https://dl.acm.org/ccs/ccs_flat.cfm 

\keywords{\ISA, ISA specification, Formal Semantics}%TODO mandatory; please add comma-separated list of keywords

\category{}%optional, e.g. invited paper

\relatedversion{}%optional, e.g. full version hosted on arXiv, HAL, or other respository/website
%\relatedversion{A full version of the paper is available at \url{...}.}

\supplement{}%optional, e.g. related research data, source code, ... hosted on a repository like zenodo, figshare, GitHub, ...

%\funding{(Optional) general funding statement \dots}%optional, to capture a funding statement, which applies to all authors. Please enter author specific funding statements as fifth argument of the \author macro.

%\acknowledgements{I want to thank \dots}%optional

%\nolinenumbers %uncomment to disable line numbering

%\hideLIPIcs  %uncomment to remove references to LIPIcs series (logo, DOI, ...), e.g. when preparing a pre-final version to be uploaded to arXiv or another public repository

%Editor-only macros:: begin (do not touch as author)%%%%%%%%%%%%%%%%%%%%%%%%%%%%%%%%%%
\EventEditors{John Q. Open and Joan R. Access}
\EventNoEds{2}
\EventLongTitle{42nd Conference on Very Important Topics (CVIT 2016)}
\EventShortTitle{CVIT 2016}
\EventAcronym{CVIT}
\EventYear{2016}
\EventDate{December 24--27, 2016}
\EventLocation{Little Whinging, United Kingdom}
\EventLogo{}
\SeriesVolume{42}
\ArticleNo{23}
%%%%%%%%%%%%%%%%%%%%%%%%%%%%%%%%%%%%%%%%%%%%%%%%%%%%%%

\input{custom_package.tex}

\begin{document}

\maketitle

%TODO mandatory: add short abstract of the document
\begin{abstract}
    The \ISA instruction set architecture (ISA) is one of the
    most complex and widely used ISAs on servers and desktops,
    and ensuring the correctness of the x86-64 binary code is
    important. A formal semantics of \ISA is required for formal reasoning
    about binary code which is one of the strongest ways to ensure
    its correctness. 
    %Completely formalizing
    %the semantics of x86-64, however, is extremely challenging
    %especially due to the complexity and the sheer number of
    %instructions that are informally specified, using a mixture of prose and pseudocode, in thousands of pages in the \emph{Intel reference manual}. 
    %%
    %Despite the challenges, we formalized, in our previous work, the semantics of most of the user-level instruction using language semantic engineering framework \K. 
    In this work, we present the ongoing work on augmenting the capability of the formalized semantics by formalizing the instruction decoder in \K. Also, we highlight  the engineering effort in improving the performance of the interpreter derived right from the semantics. 
    \revisit{TODO}
\end{abstract}

\section{Introduction}
\label{sec:intro}
x86-64 being one of the most complex ISAs and widely used in servers and desktop, ensuring the correctness of the program written in binary is extremely important. A formal semantics of x86-64 is imperative for formal reasoning about binary code which is  one of the strongest ways to ensure it's correctness. A formal semantic of the ISA allows direct reasoning about the binary code which is at times more desirable than the source level analysis. For example, in case of legacy software, when the source code is missing, or malware analysis (e.g. finding CFI attacks using gadgets), the direct binary analysis seems to be the only viable option.  Also, often times it is not desirable to add the compiler in the trusted base because, due to presence of compiler bugs or some aggressive optimization that compiler does in the presence of undefined behaviour, there might not be any guarantee that the analysis results that we gather at the source level  can be applied faithfully to the binary code.  

Completely formalizing the semantics of x86-64, however, is extremely challenging especially due to the complexity and the sheer number of instructions that are informally specified, using a mixture of prose and pseudocode, in thousands of pages in the \emph{Intel reference manual}. 
%%
Despite the challenges, we witnessed the heroic effort~\cite{Dasgupta:2019} in formalizing the semantics of most of the user-level instruction using language semantic engineering framework \K. However, the developed semantics is for the assembly language notation of the binary program and limited in the sense that any binary analysis using that formalism require, as a prerequisite, converting the binary to the supported mnemonic notation using an off-the-shelf disassembler. Such a requirement add the disassembler in the trusted computing base.  


%%
%% Bibliography
%%

%% Please use bibtex, 

\bibliography{bibs/references.bib,bibs/modeling-X86-semantics.bib,bibs/bugs.bib,bibs/k.bib}

\end{document}
