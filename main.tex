
\documentclass[a4paper,UKenglish,cleveref, autoref]{lipics-v2019}
%This is a template for producing LIPIcs articles. 
%See lipics-manual.pdf for further information.
%for A4 paper format use option "a4paper", for US-letter use option "letterpaper"
%for british hyphenation rules use option "UKenglish", for american hyphenation rules use option "USenglish"
%for section-numbered lemmas etc., use "numberwithinsect"
%for enabling cleveref support, use "cleveref"
%for enabling cleveref support, use "autoref"


%\graphicspath{{./graphics/}}%helpful if your graphic files are in another directory

\bibliographystyle{plainurl}% the mandatory bibstyle

\title{Formalizing \ISA Instruction Decoder in \K} %TODO Please add

\titlerunning{x86-64 ISA semantics}%optional, please use if title is longer than one line

%\author{Sandeep Dasgupta}{University of Illinois at Urbana Champaign, USA \and \url{http://sdasgup3.web.engr.illinois.edu}}{sdasgup3@illinois.edu}{https://orcid.org/0000-0002-6483-2055}%TODO mandatory, please use full name; only 1 author per \author macro; first two parameters are mandatory, other parameters can be empty. Please provide at least the name of the affiliation and the country. The full address is optional

%\author{Sandeep Dasgupta}{University of Illinois at Urbana Champaign, USA \and \url{http://sdasgup3.web.engr.illinois.edu} }{sdasgup3@illinois.edu}{https://orcid.org/0000-0002-6483-2055}{}
\author{Sandeep Dasgupta}{University of Illinois at Urbana Champaign, USA \and \url{http://sdasgup3.web.engr.illinois.edu} }{sdasgup3@illinois.edu}{}{}
\author{Andrew H. Miranti}{University of Illinois at Urbana Champaign, USA}{miranti2@illinois.edu}{}{}


\authorrunning{S. Dasgupta and A.H. Miranti}%TODO mandatory. First: Use abbreviated first/middle names. Second (only in severe cases): Use first author plus 'et al.'

\Copyright{Sandeep Dasgupta and Andrew H. Miranti}%TODO mandatory, please use full first names. LIPIcs license is "CC-BY";  http://creativecommons.org/licenses/by/3.0/

\ccsdesc[100]{General and reference~General literature}
\ccsdesc[100]{General and reference}%TODO mandatory: Please choose ACM 2012 classifications from https://dl.acm.org/ccs/ccs_flat.cfm 

\keywords{\ISA, ISA specification, Formal Semantics}%TODO mandatory; please add comma-separated list of keywords

\category{}%optional, e.g. invited paper

\relatedversion{}%optional, e.g. full version hosted on arXiv, HAL, or other respository/website
%\relatedversion{A full version of the paper is available at \url{...}.}

\supplement{}%optional, e.g. related research data, source code, ... hosted on a repository like zenodo, figshare, GitHub, ...

%\funding{(Optional) general funding statement \dots}%optional, to capture a funding statement, which applies to all authors. Please enter author specific funding statements as fifth argument of the \author macro.

%\acknowledgements{I want to thank \dots}%optional

%\nolinenumbers %uncomment to disable line numbering

%\hideLIPIcs  %uncomment to remove references to LIPIcs series (logo, DOI, ...), e.g. when preparing a pre-final version to be uploaded to arXiv or another public repository

%Editor-only macros:: begin (do not touch as author)%%%%%%%%%%%%%%%%%%%%%%%%%%%%%%%%%%
\EventEditors{John Q. Open and Joan R. Access}
\EventNoEds{2}
\EventLongTitle{42nd Conference on Very Important Topics (CVIT 2016)}
\EventShortTitle{CVIT 2016}
\EventAcronym{CVIT}
\EventYear{2016}
\EventDate{December 24--27, 2016}
\EventLocation{Little Whinging, United Kingdom}
\EventLogo{}
\SeriesVolume{42}
\ArticleNo{23}
%%%%%%%%%%%%%%%%%%%%%%%%%%%%%%%%%%%%%%%%%%%%%%%%%%%%%%

%% Custom Packages
\usepackage[T1]{fontenc}
\usepackage{listings}
\usepackage{color}
\usepackage{xspace}
\usepackage[title]{appendix}
%\usepackage{tikz}
%\usepackage{pgf-pie}
\usepackage{forest}
\usepackage{amsmath,amssymb}
%\usepackage{ctable}
\usepackage{pifont}
\usepackage{calculator}
\usepackage[utf8]{inputenc} 
\usepackage[T1]{fontenc}
\usepackage{microtype}
% balance
\usepackage{flushend}
\usepackage{balance}

% opphans
\clubpenalty = 10000
\widowpenalty = 10000
\displaywidowpenalty = 10000

%\setlength{\parskip}{0.5pt plus 4pt minus 3pt}
%\setlength{\textfloatsep}{1\baselineskip plus 0.2\baselineskip minus 0.5\baselineskip}
\newenvironment{tightcenter}{%
    \setlength\topsep{4pt}
    \setlength\parskip{-2pt}
    \begin{center}
    }{%
    \end{center}
}

% FIXME: overleaf broken if uncommented
% \usepackage{tikz-qtree}
% \usetikzlibrary{arrows,shapes,positioning,shadows,trees}
% \usetikzlibrary{shadows,trees}


%% Custom Commands
\def\Code#1{\texttt{#1} }
%\def\Comment#1{}
\def\Comment#1{\textbf{\textsl{\color{red}  $\langle\!\langle$#1$\rangle\!\rangle$}} }
\newcommand{\percentage}[2]{\DIVIDE{#1}{#2}{\duv}\MULTIPLY{\div}{100}{\res}$\res\%$}
\newcommand{\revisit}[1]{{\color{red} Sandeep: #1}}
\newcommand{\Qd}[1]{{\color{red} Daejun: #1}}
\newcommand{\Qt}[1]{{\color{red} Theo: #1}}
\newcommand{\BW}[1]{{\color{red} Borrowed: #1}}
\newcommand{\Added}[1]{{\color{red} #1}}
%\newcommand{\SC}[1]{{\color{blue} #1}}
%\newcommand{\AEC}[1]{{\color{blue} #1}}
\newcommand{\SC}[1]{#1}
\newcommand{\AEC}[1]{#1}
\newcommand{\cmt}[1]{}
\newcommand{\xmark}{{\color{red} \ding{55}}}
\newcommand{\cmark}{{\color{green} \ding{51}}}
\newcommand{\ISA}{x86-64\xspace}
\newcommand{\GENISA}{x86\xspace}
%\newcommand{\K}{\mbox{$\mathbb{K}$}\xspace}
\newcommand{\Z}{$\mathbb{Z}3$\xspace}
\newcommand{\uif}{uninterpreted functions}
\newcommand{\Strata}{Strata\xspace}
\newcommand{\Stoke}{Stoke\xspace}
\newcommand{\initS}{{\tt initial search}}
\newcommand{\secS}{{\tt secondary searches}}
%\newcommand{\K}{\ensuremath{\mathcal{{\tt K}}}\xspace}
\newcommand{\TS}[1]{{\tt #1}}
%\newcommand{\instr}[1]{\texttt{#1}}
\newcommand{\instr}[1]{\textbf{\color{brown}\m{#1}}}
\newcommand{\reg}[1]{\s{\%#1}}
\newcommand{\mem}[2]{\s{#1(\%#2)}}
\newcommand{\opcode}[1]{\ensuremath{#1}}
%%
\newcommand{\Op}{\s{Opcode}\xspace}
\newcommand{\MODRM}{\s{MODRM}\xspace}
\newcommand{\SIB}{\s{SIB}\xspace}
\newcommand{\Disp}{\s{Displacement}\xspace}
%\newcommand{\cond}[1]{\ensuremath{#1}}
\newcommand{\extract}{\emph{extract}\xspace}
\newcommand{\extractMInt}{\emph{extractMInt}\xspace}
\newcommand{\false}{\textbf{False}}
\newcommand{\true}{\textbf{True}}
\newcommand{\bool}{\texttt{Bool}\xspace}
\newcommand{\incfig}[1]{\includegraphics[scale=.7]{#1}}
\newcommand{\CF}[2]{$\s{F}_{\s{#2}}^{\s{#1}}$}
\newcommand{\GN}[2]{$G{[#2]}^{#1}$}
\newcommand{\udef}{\emph{undef}\xspace}
%\newcommand{\bv}[2]{$#1\text{'}#2$\xspace}
\newcommand{\bv}[2]{\s{#1}\text{'}\s{#2}}

\newcommand{\rating}[1]{%
    \begin{tikzpicture}[x=1ex,y=1ex]
    \begin{scope}
    \clip (0,1) circle (1);
    \fill[black] (-1,0) rectangle (1,#1/50);
    \end{scope}
    \draw[black, thin, radius=1] (0,1) circle;
    \end{tikzpicture}%
}

% Current Support
%\newcommand{\currentIS}{$3186$} %including jmp label
\newcommand{\currentIS}{$3155$}
\newcommand{\currentIntel}{$774$}
%\newcommand{\currentManual}{$1199$} %3104 - 1905
%\newcommand{\currentManual}{$1281$} %3186 - 1905
%\newcommand{\currentManualPerc}{$40\%$} %100 - \strataPerc
% Total
\newcommand{\totalIS}{$3736$}
\newcommand{\totalIntel}{$996$}
%\newcommand{\totalIntel}{$1,000$}
\newcommand{\dup}{$109$}
% Strata
\newcommand{\strataIS}{$1796$}
\newcommand{\strataIntel}{$466$}
\newcommand{\strataWithDupIS}{$1905$}
\newcommand{\strataRegVarIS}{$692$}
% Unsupported
\newcommand{\system}{$210$}
\newcommand{\Xmmx}{$336$}
\newcommand{\crypto}{$35$}

\newcommand{\strataPerc}{$47\%$} % 466/996 or  1905 / 3736
\newcommand{\goelPerc}{$33\%$} 
\newcommand{\sailPerc}{$15\%$} 
% Stoke disjoin from Strata
%\newcommand{\stokeIS}{$332$} % 262 + 15 + 9 + 46. ALso 332/3767 == 9%
% 1432(strata common) + 332
\newcommand{\stokeIS}{${\sim}1764$}
%\newcommand{\stokeExcPerc}{$9\%$}
% Strata stoke combined
%\newcommand{\strataPlusStokeIS}{$2237$} % 

%\newcommand{\unsupp}{$939$}

%%%%%% Immediates
%\newcommand{\ImmUg}{$146$} % 118 + 28
%\newcommand{\ImmTotal}{$308$}
%\newcommand{\ImmG}{$190$}

%%%%%%% Registers
%\newcommand{\RegTOTAL}{$1133$} % 1083 + 50
%\newcommand{\RegSTRAT}{$742$} % 692 + 50
%\newcommand{\RegSTOK}{$262$}
%\newcommand{\RegMAN}{$129$}

%%% toture status
\newcommand{\TortureTotal}{$1576$} %
\newcommand{\TortureExclude}{$6$} % 6 + 22
\newcommand{\TortureInclude}{$1548$}
\newcommand{\TortureUifsInstr}{$293$} % 134(all three jobs) +  48
\newcommand{\TortureUifs}{$35$}
\newcommand{\TortureCoverage}{$963$}
%%% Undef counts
\newcommand{\undefTotal}{$474$}
\newcommand{\undefIntel}{$32$}
\newcommand{\undefPerc}{$3$} %32/1000

\PassOptionsToPackage{pdftex,usenames,dvipsnames,svgnames,x11names}{xcolor}
\PassOptionsToPackage{pdftex}{hyperref}
\usepackage[style=math]{k}


% Slightli modified original version of \reduce. Altered baseline for more compactness.
% No support for multiline.
\newcommand{\reduceClassic}[2]{\hbox{%
  \begin{tikzpicture}[baseline=(top.south), %(top.base), - default, less compact
                      inner xsep=0pt,
                      inner ysep=.3333ex,
                      minimum width=2em]
    \path
          % Original version. No support for line wrapping.
          node (top) [inner ysep=1ex]{$#1$ \mathstrut}

          % New version. Line wrapping support.
          %node (top) [inner ysep=1ex]{$ \begin{array}{@{}c@{}} #1 \end{array} $ \mathstrut}
          (top.south)
          % Original version. No support for line wrapping.
          node (bottom) [anchor=north, inner ysep=.5ex] {$#2$};

          % New version. Line wrapping support.
          % Adds a little bit of vertical space, but the difference is truly insignificant. All the experiments below failed to remove it.
          %node (bottom) [anchor=north, inner ysep=.5ex] {$ \begin{array}{@{}c@{}} #2 \end{array} $};
          % no extra effect
          % node (bottom) [anchor=north, inner ysep=.5ex] {\vspace{-1em} $ \begin{array}{@{}c@{}} #2 \end{array} $};
          % trying mathstrut - some horizontal re-alignment, but no vertical
          % node (bottom) [anchor=north, inner ysep=.5ex] {$ \begin{array}{@{}c@{}} #2 \end{array} $ \mathstrut};
          % no outer ysep (if no inner - looks bad)
          %node (bottom) [anchor=north, inner ysep=.5ex, outer ysep=0] {$ \begin{array}{@{}c@{}} #2 \end{array} $};
          % \vskip -1em just don't compile no matter where we put it
    \path[draw,thin,solid] let \p1 = (current bounding box.west),
                               \p2 = (current bounding box.east),
                               \p3 = (top.south)
                           in (\x1,\y3) -- (\x2,\y3);
    % Solid arrow (augmenting the solid line).
    \path[fill] (top.south) ++(2pt,0) -- ++(-4pt,0) -- ++(2pt,-1.5pt) -- cycle;
  \end{tikzpicture}%
}}

% Defalut version of \reduce in this document.
%   Support for multi-line LHS and RHS
%   Good compactness. Separators adjusted to be aligned with \reduceClassic
\newcommand{\reduceMulti}[2]{\hbox{%
  \begin{tikzpicture}[baseline=(top.south), %(top.base), - default, less compact
                      inner xsep=0pt,
                      inner ysep=.3333ex,
                      minimum width=2em]
    \path
          % New version. Line wrapping support.
          node (top) [
            %inner ysep=1ex
            inner ysep=0.6ex
          ]{ $ \begin{array}{@{}c@{}}
                #1
               \end{array} $ \mathstrut}
          (top.south)
          % New version. Line wrapping support.
          node (bottom) [
            anchor=north,
            %inner ysep=.5ex
          ] {
            $ \begin{array}{@{}c@{}}
              #2
            \end{array} $};
    \path[draw,thin,solid] let \p1 = (current bounding box.west),
                               \p2 = (current bounding box.east),
                               \p3 = (top.south)
                           in (\x1,\y3) -- (\x2,\y3);
    % Solid arrow (augmenting the solid line).
    \path[fill] (top.south) ++(2pt,0) -- ++(-4pt,0) -- ++(2pt,-1.5pt) -- cycle;
  \end{tikzpicture}%
}}

%Special version of \reduce with modified baseline, for better rendering of multiline
% LHS and RHS
\newcommand{\reduceCompact}[2]{\hbox{%
  \begin{tikzpicture}[baseline=(bottom), %(top.base), - default, less compact
                      inner xsep=0pt,
                      inner ysep=.3333ex,
                      minimum width=2em]
    \path
          node (top) [inner ysep=0.6ex]{$ \begin{array}{@{}c@{}} #1 \end{array} $ \mathstrut}
          (top.south)
          node (bottom) [anchor=north] {$ \begin{array}{@{}c@{}} #2 \end{array} $};
    \path[draw,thin,solid] let \p1 = (current bounding box.west),
                               \p2 = (current bounding box.east),
                               \p3 = (top.south)
                           in (\x1,\y3) -- (\x2,\y3);
    % Solid arrow (augmenting the solid line).
    \path[fill] (top.south) ++(2pt,0) -- ++(-4pt,0) -- ++(2pt,-1.5pt) -- cycle;
  \end{tikzpicture}%
}}

%\renewcommand{\reduce}[2]{\reduceClassic{#1}{#2}}
\renewcommand{\reduce}[2]{\reduceMulti{#1}{#2}}




%\lstset{captionpos=t,tabsize=3,frame=no,keywordstyle=\color{blue},
%        commentstyle=\color{gray},stringstyle=\color{red},
%        breaklines=true,showstringspaces=false,emph={label},
%        basicstyle=\ttfamily}

% Required in order to make \kall cells inside comments black.
\renewcommand{\kall}[3][black]{\mall{#1}{#2}{#3}}

% Environment "kdefinition" has effect only in poster style, thus in math style may be safely deleted.

%Continuation of a syntax definition on a new line
\newcommand{\syntaxContNewLine}[3][\defSort]{\par\indent\rulebox{%
  $\setlength{\syntaxlength}{\widthof{$\mathrel{::=}$}}%
  \setlength{\syntaxlength}{.5\syntaxlength}%
  \addtolength{\syntaxlength}{\widthof{\syntaxKeyword$#1$}}%
  \hspace{\syntaxlength}%
  \;\;\;\;\;\;\;{#2}$ \ifthenelse{\equal{#3}{}}{}{[#3]}%
  }%\k@markPosition%
}

%Should be put after a syntaxLong.
\newcommand{\syntaxEnd}[3][\defSort]{
  \indent\rulebox{%
  $\setlength{\syntaxlength}{\widthof{$\mathrel{::=}$}}%
  \setlength{\syntaxlength}{.5\syntaxlength}%
  \addtolength{\syntaxlength}{\widthof{\syntaxKeyword$#1$}}%
  \hspace{\syntaxlength}$%
  }%\k@markPosition%
}

\newcommand{\syntaxLong}[3][\defSort]{\rulebox{%
\syntaxKeyword
$
  \begin{array}[t]{@{}l@{}}
  #1 \\
  \mathrel{::=}{#2}
  \end{array}
$ {}%
}%\k@markPosition%
}

% Grigore's idea macro
\newcommand{\idea}[1]{
  \begin{quote}
    \rule{.45\textwidth}{.5pt}\newline
    {\em #1}
    \vspace*{-1ex}\newline \rule{.45\textwidth}{.5pt}
  \end{quote}
}

\newenvironment{ideas}
{ \begin{quote}
    \rule{.45\textwidth}{.5pt}
    \newline
    \begin{em}
} {
    \end{em}
    \vspace*{-1ex}
    \leavevmode
    \newline
    \rule{.45\textwidth}{.5pt}
  \end{quote}
}

%Enforcing black cells
\renewcommand{\kall}[3][white]{\mall{black}{#2}{#3}}
\renewcommand{\kallLarge}[3][white]{\mallLarge{black}{#2}{#3}}
\renewcommand{\kprefix}[3][white]{\mprefix{black}{#2}{#3}}
\renewcommand{\ksuffix}[3][white]{\msuffix{black}{#2}{#3}}
\renewcommand{\kmiddle}[3][white]{\mmiddle{black}{#2}{#3}}

% Settigns required for Chucky's background section
\usepackage{acronym}

\providecommand{\Sec}{}
\renewcommand{\Sec}{Section~}
\newcommand{\Fig}{Figure~}

\newcommand{\cellname}[1]{\textsf{#1}}

\newcommand{\kequation}[2]{\begin{equation*}{\small#2}\end{equation*}}

%Probably a mapsto with spacing
\newcommand{\mapstox}{\small\mathrel{\mapsto}}

%For spacing between cell lines
%\newcommand{\kBR}{\\[0.3em]}

% General
\newcommand{\w}[1]{\ensuremath{\textit{#1}}}
\newcommand{\m}[1]{\ensuremath{\texttt{#1}}}
\newcommand{\s}[1]{\ensuremath{\textsf{#1}}}
\newcommand{\p}[1]{\ensuremath{\left(#1\right)}}
\newcommand{\pl}[1]{\ensuremath{\left\langle#1\right\rangle}}
\newcommand{\OR}{\mbox{ }|\mbox{ }}
%\newcommand{\st}{.\mbox{ }}
\newcommand{\finto}{\ensuremath{\stackrel{\mathtt{fin}}{\longrightarrow}}}
\newcommand{\defeq}{\ensuremath{\stackrel{\mathtt{def}}{=}}}
\newcommand{\cond}[1]{\ensuremath{\left\{\begin{array}{ll} #1 \end{array}\right.}}
\newcommand{\lst}[1]{\begin{itemize} {#1} \end{itemize}}
\newcommand{\pby}[1]{\hspace*{\fill}{#1}}
\newcommand{\slide}[2][]{ \begin{frame} \frametitle{#1} {#2} \end{frame} }
\newcommand{\etal}{\textit{et~al.}\xspace}

% For references
\newcommand{\fig}[1]{Figure~\ref{#1}}
\newcommand{\lem}[1]{Lemma~\ref{#1}}
\newcommand{\theo}[1]{Theorem~\ref{#1}}
\newcommand{\coro}[1]{Corollary~\ref{#1}}
\newcommand{\defn}[1]{Definition~\ref{#1}}
\newcommand{\rmrk}[1]{Remark~\ref{#1}}
\newcommand{\exam}[1]{Example~\ref{#1}}
\newcommand{\sect}[1]{$\S$~\ref{#1}}

%\newcommand{\todo}[1]{}
%\newcommand{\todo}[1]{{\textcolor{red}{\textbf{[[{#1}]]}}}}

\usepackage{ucs} % for unicode characters (just for \Rosu and \Serbanuta)
\newcommand{\sh}{\unichar{0537}}
\newcommand{\Sh}{\unichar{0536}}
\PrerenderUnicode{\sh}
\PrerenderUnicode{\Sh}
\newcommand{\Rosu}{Ro{\sh}u\xspace}
\newcommand{\Stefanescu}{{\Sh}tef{\u a}nescu\xspace}

\newcommand{\JS}{JavaScript\xspace}
\newcommand{\ES}{ECMAScript\xspace}
%\newcommand{\K}{\ensuremath{\mathbb{K}}\xspace}
%\newcommand{\KJS}{\ensuremath{\mathbb{K}}JS\xspace}
\newcommand{\KJS}{KJS\xspace}
\newcommand{\LJS}{\ensuremath{\lambda_{\w{JS}}}\xspace}
\newcommand{\spec}{specification\xspace}

\lstdefinelanguage{JavaScript}{
  keywords={break, case, catch, continue, debugger, default, delete, do, else, finally, for, function, if, in, instanceof, new, return, switch, this, throw, try, typeof, var, void, while, with},
  morecomment=[l]{//},
  morecomment=[s]{/*}{*/},
  morestring=[b]',
  morestring=[b]",
  sensitive=true
}

\definecolor{orange}{rgb}{1,0.5,0}
\definecolor{darkgreen}{rgb}{0.0, 0.5, 0.0}
%\newcommand{\note}[2]{\textbf{\textit{\textcolor{#1}{[[{#2}]]}}}}
\newcommand{\marker}[1]{} %{\note{orange}{{#1}}}
\newcommand{\daejun}[1]{\note{red}{Daejun: {#1}}}
\newcommand{\andrei}[1]{\note{darkgreen}{Andrei: {#1}}}
\newcommand{\grigore}[1]{\note{blue}{Grigore: {#1}}}



\definecolor{codegreen}{rgb}{0,0.6,0}
\definecolor{codegray}{rgb}{0.5,0.5,0.5}
\definecolor{codepurple}{rgb}{0.58,0,0.82}
\definecolor{backcolour}{rgb}{0.95,0.95,0.92}


\lstdefinestyle{Bash}{
    language=Bash,                % choose the language of the code
    basicstyle=\footnotesize,       % the size of the fonts that are used for the code
    numbers=left,                   % where to put the line-numbers
    numberstyle=\tiny\color{codegray},      % the size of the fonts that are used for the line-numbers
    stepnumber=1,                   % the step between two line-numbers. If it is 1 each line will be numbered
    numbersep=5pt,                  % how far the line-numbers are from the code
    backgroundcolor=\color{white},  % choose the background color. You must add \usepackage{color}
    showspaces=false,               % show spaces adding particular underscores
    showstringspaces=false,         % underline spaces within strings
    showtabs=false,                 % show tabs within strings adding particular underscores
    frame=single,           % adds a frame around the code
    %tabsize=2,          % sets default tabsize to 2 spaces
    captionpos=b,           % sets the caption-position to bottom
    breaklines=true,        % sets automatic line breaking
    breakatwhitespace=false,    % sets if automatic breaks should only happen at whitespace
    escapeinside={\%*}{*)},          % if you want to add a comment within your code
    commentstyle=\color{gray},
    keywordstyle=\color{blue},
    morekeywords={andnq, jp, jz, movw, movq, xorq, orq, retq, pushw}
}

\lstdefinestyle{C++}{
    language=C++,                % choose the language of the code
    basicstyle=\footnotesize,       % the size of the fonts that are used for the code
    numbers=left,                   % where to put the line-numbers
    numberstyle=\tiny\color{codegray},      % the size of the fonts that are used for the line-numbers
    stepnumber=1,                   % the step between two line-numbers. If it is 1 each line will be numbered
    numbersep=5pt,                  % how far the line-numbers are from the code
    backgroundcolor=\color{white},  % choose the background color. You must add \usepackage{color}
    showspaces=false,               % show spaces adding particular underscores
    showstringspaces=false,         % underline spaces within strings
    showtabs=false,                 % show tabs within strings adding particular underscores
    frame=single,           % adds a frame around the code
    %tabsize=2,          % sets default tabsize to 2 spaces
    captionpos=b,           % sets the caption-position to bottom
    breaklines=true,        % sets automatic line breaking
    breakatwhitespace=false,    % sets if automatic breaks should only happen at whitespace
    escapeinside={\%*}{*)},          % if you want to add a comment within your code
    commentstyle=\color{gray},
    keywordstyle=\color{blue},
}

\lstdefinestyle{SMTLIB}{
    language=Java,
    basicstyle=\footnotesize,       % the size of the fonts that are used for the code
    numbers=left,                   % where to put the line-numbers
    numberstyle=\tiny\color{codegray},      % the size of the fonts that are used for the line-numbers
    stepnumber=1,                   % the step between two line-numbers. If it is 1 each line will be numbered
    numbersep=5pt,                  % how far the line-numbers are from the code
    backgroundcolor=\color{white},  % choose the background color. You must add \usepackage{color}
    showspaces=false,               % show spaces adding particular underscores
    showstringspaces=false,         % underline spaces within strings
    showtabs=false,                 % show tabs within strings adding particular underscores
    frame=single,           % adds a frame around the code
    %tabsize=2,          % sets default tabsize to 2 spaces
    captionpos=b,           % sets the caption-position to bottom
    breaklines=true,        % sets automatic line breaking
    breakatwhitespace=false,    % sets if automatic breaks should only happen at whitespace
    escapeinside={(*}{*)},          % if you want to add a comment within your code
    commentstyle=\color{gray},
    keywordstyle=\color{blue},
    morekeywords={bvand, bvnot, concat, extract, bvxor}
}

\lstdefinestyle{KRULE}{
    %language=Java,
    %basicstyle=\footnotesize,      
    basicstyle=\scriptsize,
    backgroundcolor=\color{white},  % choose the background color. You must add \usepackage{color}
    showspaces=false,               % show spaces adding particular underscores
    showstringspaces=false,         % underline spaces within strings
    showtabs=false,                 % show tabs within strings adding particular underscores
    %frame=single,           % adds a frame around the code
    %tabsize=2,          % sets default tabsize to 2 spaces
    captionpos=b,           % sets the caption-position to bottom
    breaklines=true,        % sets automatic line breaking
    breakatwhitespace=false,    % sets if automatic breaks should only happen at whitespace
    escapeinside={(*}{*)},          % if you want to add a comment within your code
    commentstyle=\color{gray},
    morecomment=[l]{//},
    keywordstyle=\color{blue},
    %morekeywords={regstate, stackmem, andBool, requires, ensures, codemem, memstate, and}
    morekeywords={andBool, requires, ensures, and, rule}
}

\lstdefinestyle{KRULEWOBORDER}{
    %language=Java,
    %basicstyle=\footnotesize,      
    basicstyle=\scriptsize,
    backgroundcolor=\color{white},  % choose the background color. You must add \usepackage{color}
    showspaces=false,               % show spaces adding particular underscores
    showstringspaces=false,         % underline spaces within strings
    showtabs=false,                 % show tabs within strings adding particular underscores
    %frame=single,           % adds a frame around the code
    %tabsize=2,          % sets default tabsize to 2 spaces
    captionpos=b,           % sets the caption-position to bottom
    breaklines=true,        % sets automatic line breaking
    breakatwhitespace=false,    % sets if automatic breaks should only happen at whitespace
    escapeinside={(*}{*)},          % if you want to add a comment within your code
    commentstyle=\color{gray},
    morecomment=[l]{//},
    keywordstyle=\color{blue},
    morekeywords={regstate, stackmem, andBool, requires, ensures, codemem, memstate, and}
}

\lstdefinestyle{SIMPRULES}{
    language=Java,
    basicstyle=\footnotesize,       % the size of the fonts that are used for the code
    backgroundcolor=\color{white},  % choose the background color. You must add \usepackage{color}
    escapeinside={(*}{*)},          % if you want to add a comment within your code
    commentstyle=\color{gray},
    morecomment=[l]{//},
}


\begin{document}

\maketitle

%TODO mandatory: add short abstract of the document
\begin{abstract}
    The \ISA instruction set architecture (ISA) is one of the
    most complex and widely used ISAs on servers and desktops,
    and ensuring the correctness of the x86-64 binary code is
    important. A formal semantics of \ISA is required for formal reasoning
    about binary code which is one of the strongest ways to ensure
    its correctness. 
    %Completely formalizing
    %the semantics of x86-64, however, is extremely challenging
    %especially due to the complexity and the sheer number of
    %instructions that are informally specified, using a mixture of prose and pseudocode, in thousands of pages in the \emph{Intel reference manual}. 
    %%
    %Despite the challenges, we formalized, in our previous work, the semantics of most of the user-level instruction using language semantic engineering framework \K. 
    In this work, we present the ongoing work on augmenting the capability of the formalized semantics by formalizing the instruction decoder in \K. Also, we highlight  the engineering effort in improving the performance of the interpreter derived right from the semantics. 
    \revisit{TODO}
\end{abstract}

\section{Introduction}
\label{sec:intro}
x86-64 being one of the most complex ISAs and widely used in servers and desktop, ensuring the correctness of the program written in binary is extremely important. A formal semantics of x86-64 is imperative for formal reasoning about binary code which is  one of the strongest ways to ensure it's correctness. A formal semantic of the ISA allows direct reasoning about the binary code which is at times more desirable than the source level analysis. For example, in case of legacy software, when the source code is missing, or malware analysis (e.g. finding CFI attacks using gadgets), the direct binary analysis seems to be the only viable option.  Also, often times it is not desirable to add the compiler in the trusted base because, due to presence of compiler bugs or some aggressive optimization that compiler does in the presence of undefined behaviour, there might not be any guarantee that the analysis results that we gather at the source level  can be applied faithfully to the binary code.  

Completely formalizing the semantics of x86-64, however, is extremely challenging especially due to the complexity and the sheer number of instructions that are informally specified, using a mixture of prose and pseudocode, in thousands of pages in the \emph{Intel reference manual}. 
%%
Despite the challenges, we witnessed the heroic effort~\cite{Dasgupta:2019} in formalizing the semantics of most of the user-level instruction using language semantic engineering framework \K. However, the developed semantics is for the assembly language notation of the binary program and limited in the sense that any binary analysis using that formalism require, as a prerequisite, converting the binary to the supported mnemonic notation using an off-the-shelf disassembler. Such a requirement add the disassembler in the trusted computing base.  

\section{Instruction Decoding}

While a semantics of \GENISA mnemonics is valuable on its own, its direct uses are limited alone.  Very few real world programs are most conveniently available in assembler form, and not source or binary forms.   Formalizations for many source code languages already exist, and thus would be a natural choice in the event that source code is available.  However, many programs must make use of precompiled binaries for which source is not available – either due to IP concerns, the circumstances of the researcher (as an example, malware researchers) or due to the age of the program.  In order to reason about these programs, one must work from what one has: an executable binary file.  Thus, in order to apply the existing semantics of \GENISA to more common real world programs, the semantics must incorporate a disassembler of some form to translate from binary back to the modeled assembly language.
\subsection{Approach}
Two basic approaches to applying binaries to the existing x86 semantics present themselves.  First, one could disassemble the entire binary into a set of x86 assembler files, and then pass those directly to run on the semantics unmodified.  Alternatively, one could formalize the semantics of instruction decoding, and perform the decoding at runtime with the semantics.  The first option would seem relatively simpler, as it would allow the use of existing tools to perform the decoding.  However, disassembly of x86 programs is no trivial task.  While there exist many tools capable of disassembling a binary, they are forced to make certain assumptions about the programs under disassembly.  The basic algorithms are as follows:

\begin{enumerate}
    \item Linear sweep: Starting from the beginning of the text segment, disassemble instructions sequentially over the entire program.  This approach is the simplest, but breaks if it starts at an incorrect offset, or unexpected data appears in the text segment of the program.
    \item Recursive descent: Starting at a known address (e.g. the program entry), decode sequentially until a branch is detected, in which case spawn another thread which decodes the target.
    \item Probabilistic~\cite{Miller:2019}: Computes a probability for each address in the code space, indicating its likelihood of being a branch target. 
\end{enumerate}

However, each of these algorithms share a common flaw: they cannot offer any certainty that the decoded program is actually the one that would run if the binary were executed, and thus conclusions drawn from that program must be met with some skepticism.  As an example of why, consider the target of an indirect jump (of the form \instr{jmp \%rax}) – where the target is computed by an arbitrary function.  Determining the value of that function’s return can be arbitrarily complex, up to undecidable.  Thus, the location of program flow is, in the general case, undecidable.  Few, if any, real world x86 programs lack any indirect jumps, as \instr{ret} performs one. 

In order to sidestep the above issues, we took the alternative approach of decoding individual instructions as the semantics execute them.  At any given step of execution, the semantics know exactly where the program counter is, and thus exactly where the next instruction should start.  After decoding and executing the instruction, the semantics know exactly where the program counter should go next (from the size and semantics of the instruction decoded).  This however requires a formalization of instruction decoding in \K, in order to execute the decoder in tandem with the semantics.  To avoid reinventing the wheel, we chose to base the formalized decoder implementation on an existing implementation - Intel’s x86 instruction decoding and disassembly tool, XED~\cite{xed}.  We modified the algorithm to better fit the application and environment (notably, the \K programming language). 

An \GENISA instruction can be broken down into $0-4$ prefixes, the opcode byte(s), the MODRM byte, the SIB byte, the displacement and the immediate(s).  Of these sections, all but the opcodes are optional.  One cannot know how many of these sections exist, or how to interpret the values contained in them until they have decoded previous sections.  So the first step of the algorithm is to look for prefixes, especially \s{VEX} and \s{EVEX} prefixes, each of which will cause future opcode values to be interpreted differently.  The presence of other prefixes is also recorded, and using this information the opcode is found.  Using the information about opcode and prefixes, the presence or absence of the MODRM and SIB bytes are determined, and these bytes are picked apart for their constituent values.  After all this data has been extracted, the precise instruction and operand variant can be determined.  This is where the \K Framework in particular shines – matching arbitrary subsets of properties to particular patterns is the foundation of both the \K rule, and this pivotal instruction selection step.  This match gives this step an extremely intuitive representation in \K – one rule per instruction variant.  Moreover, the input data fed to the automatically generated sections of the original Intel decoder can be re-purposed to automatically generate these \K rules.  The reference decoder requires complex generated code to perform what amounts to pattern matching on various decode properties that can be known without the instruction variant (as an example, branching between two possible sets of variants on whether the MOD bits of MODRM were 0b11).  \K can do all of these pattern matches in one step, and in a far more legible format by simply incorporating the desired constraint into a rule. After instruction selection, the sizes and positions of the remaining sections of the instructions are known to the decoder, and it can extract them trivially.  After this point, the next challenge is using the decoded instruction to generate the desired semantics operation.  Unfortunately AT\&T syntax (which the most complete \ISA semantics~\cite{Dasgupta:2019}, modeled in \K, were based on) lacks clear standardization over all of x86, and its implementation by GAS~\cite{gas} differs from its implementation by XED (for example, in how it produces suffixes for certain instructions), along with changes in instruction mnemonics between input assembler code and output disassembled code makes mapping decoded instruction mnemonics (decoder output) to assembly instruction mnemonics (semantics input) nontrivial.  This problem was eventually solved by building a lookup table of assembled instruction variants to original assembler mnemonics using GAS.  Once this step was completed, translating the instruction operands was relatively simple.

\subsection{Evaluation}
At first, the decoder was tested on its ability to  convert any binary sequence, specifying to a valid instruction in \ISA, to the corresponding mnemonic. The binary sequences are  obtained either from XED test-suite or created manually using GAS~\cite{gas} assembler. A successful completion of this experiment ensured that the decoding logic is correctly implemented and works for all the valid \ISA instructions. To gain more confidence, the decoder was also tested by combining the instruction disassembler with a simple linear sweep algorithm and then comparing it with the output of XED.  Once an acceptable accuracy was reached, the decoder was combined with the x86 semantics and run on a selection of the gcc-c torture tests~\cite{CTORTURE}. The tests were modified slightly from their original implementations by replacing certain standard library implementations with simpler versions.  This was done for the sake of feasibility, as the semantics run much more slowly than native code.  As a ground truth, a similarly modifed version of the gcc torture tests were executed normally, and any that did not pass running natively due to the simplified stdlib implementations were removed from the selection.  The gcc torture tests seem generally to be structured such that the test case will either crash or call abort in the event of an error.  This permitted us to avoid needing to compare process memory images to semantics memory images (though such a comparison would doubtlessly be useful - we leave it to future work).  A test was considered to run successfully if the semantics produced an $exit_0$ symbol on completion.  Of the $482$ of the selected torture tests that executed successfully natively, $421$ ($87.3\%$) executed successfully under the semantics.  The failures were a mix of programs that used unsupported instructions, programs which took too long and timed out, programs that called abort and programs that crashed during execution.  Though not all passed, the vast majority of the tests did (and, doubtlessly a longer timeout would have allowed several of the failure cases to pass), and continuing work will see this number increase.

\subsection{Limitation}
The new decode $+$ execution semantics is much more practically useful than the original semantics, but still has some limitations.  Dynamic linking is not supported, so the binary must be statically linked (or the dynamic sections made unreachable).  Presently the semantics do not support system calls so, although the decoder can decode them, they cannot be executed properly.  This leads to a need for some debugging symbols, as the semantics must skip to the \emph{<main>} symbol rather than starting at the program entry point, since most libc implementations’ initialization code will always make system calls.  Future work will be addressing these shortcomings, and addressing technical incompatibilities that currently block symbolic execution over the semantics.



%%
%% Bibliography
%%

%% Please use bibtex, 

\bibliography{bibs/references.bib,bibs/modeling-X86-semantics.bib,bibs/bugs.bib,bibs/k.bib}

\end{document}
